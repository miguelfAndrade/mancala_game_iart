\documentclass[conference]{IEEEtran}
\IEEEoverridecommandlockouts
% The preceding line is only needed to identify funding in the first footnote. If that is unneeded, please comment it out.
\usepackage{cite}
\usepackage{amsmath,amssymb,amsfonts}
\usepackage{algorithmic}
\usepackage{graphicx}
\usepackage{textcomp}
\usepackage{xcolor}
\def\BibTeX{{\rm B\kern-.05em{\sc i\kern-.025em b}\kern-.08em
    T\kern-.1667em\lower.7ex\hbox{E}\kern-.125emX}}
\begin{document}

\title{Trabalho 2 Vers\~ao A3\\
}

\author{\IEEEauthorblockN{Miguel Andrade (201709051)}
\IEEEauthorblockA{\textit{Faculdade de Engenharia} \\
\textit{Universidade do Porto}\\
Porto, Portugal \\
up201709051@fe.up.pt}
\and
\IEEEauthorblockN{Diana Sandoval (201811491)}
\IEEEauthorblockA{\textit{Faculdade de Engenharia} \\
\textit{Universidade do Porto}\\
Cidade do M\'exico, M\'exico \\
up201811491@fe.up.pt}
}

\maketitle

\begin{abstract}
O jogo a ser implementado ser\'a o Mancala. Começaremos por fazer uma implementaç\~ao do jogo acrescentando depois o algoritmo minimax com os cortes alpha e beta. Posteriormente acrescentaremos uma interface gr\'afica de forma a que a experi\^encia de jogo seja agrad\'avel.
\end{abstract}

\begin{IEEEkeywords}
minimax, alpha, beta, mancala, intelig\^encia artificial
\end{IEEEkeywords}

\section{Introdu\c c\~ao}

Neste segundo projeto o objetivo \'e implementar um jogo para dois jogadores, onde exista a possibilidade de haver dois jogadores humanos a jogar entre si, um jogador humano e um computador e dois computadores a jogar entre si.
Entre os jogos sugeridos, decidimos escolher o Mancala. 


\section{Descri\c c\~ao do Problema}

\begin{figure}[htbp]
    \centerline{\includegraphics[scale=0.3]{board_start_big.jpg}}
    \caption{Jogo Mancala}
    \label{img}
\end{figure}

O Mancala \'e essencialmente um jogo onde os jogadores “semeiam” e “colhem” sementes\cite{b1}.
O objetivo do jogo \'e colecionar mais sementes que o advers\'ario.
O tabuleiro de jogo \'e constituído por duas filas de 6 buracos cada uma, e na extremidades encontra-se  a mancala (dep\'osito das sementes) onde se coleciona as sementes de cada jogador.

O jogo começa com 4 sementes em cada buraco. Um jogador começa por selecionar um dos seus 6 buracos. Ao selecionar um, este coleciona todas as sementes desse buraco e começa a colocar uma semente em cada um dos buracos seguintes, no sentido contr\'ario ao ponteiros do rel\'ogio, at\'e n\~ao haver mais sementes na sua m\~ao.

Se o jogador encontrar o seu dep\'osito, este continua a jogar, se encontrar o dep\'osito do jogador advers\'ario, ent\~ao salta para o buraco seguinte. Caso seja a \'ultima semente a ser colocada no dep\'osito do pr\'oprio jogador, ent\~ao este tem direito jogar mais uma vez.
Se a \'ultima semente for colocado num buraco vazio do lado do pr\'oprio jogador, ent\~ao este pode colecionar as todas as sementes no buraco oposto, incluindo a pr\'opria semente colocada.
O jogo termina quando todos os 6 buracos de um dos lados ficam vazios. Caso um dos jogadores ainda tenha peças do seu lado quando o jogo termina, ent\~ao este coleciona-as e junta-as ao seu dep\'osito. 
Quem tiver mais sementes colecionadas \'e o vencedor.


\section{Formula\c c\~ao do Problema}

\subsection{Representação do Estado}
O Estado do jogo é representado pelo próprio tabuleiro, sendo que os depósitos dos jogadores são os fatores para decedir quem ganha.
De forma a poder representar o tabuleiro de jogo, foi usado uma lista de tamanho 14, sendo que os 7 primeiros índices dizem respeito ao primeiro jogador e os restante ao segundo jogador.
Os índices 6 e 13 são usados para representar os depósitos do primeiro e segundo jogador respetivamente. Define-se "primeiro jogador" o jogador que joga nas casas "inferiores" do tabuleiro e o "segundo jogador" o jogador que controla as casas "superiores".
A cada casa é atribuído um valor que corresponde ao número de sementes guardadas nessa casa.\\

\begin{figure}[htbp]
    \centerline{\includegraphics[scale=0.3]{tabuleiro1.jpg}}
    \caption{Tabuleiro de Jogo}
    \label{img2}
\end{figure}

\subsection{Operadores}
Cada casa jogável é um operador, sendo que as pré-condições são:
A casa escolhida não estar vazia.
A casa escolhida pertencer à área do atual jogador.

\begin{figure}[htbp]
    \centerline{\includegraphics[scale=0.3]{tabuleiro2.jpg}}
    \caption{Várias imagens a demonstrar os operadores}
    \label{img3}
\end{figure}

\subsection{Estado Inicial}
Todas as casas da lista são iniciados com o valor 4, com exceção das casas que representam os depósitos dos jogadores, sendo estas as casas 6 e 13.

\begin{figure}[htbp]
    \centerline{\includegraphics[scale=0.3]{tabuleiro3.jpg}}
    \caption{Estado Inicial}
    \label{img4}
\end{figure}


\subsection{Teste Terminal}
O estado terminal é alcançado quando um dos jogadores fica sem peças para poder continuar a jogar.\\

\begin{figure}[htbp]
    \centerline{\includegraphics[scale=0.3]{tabuleiro4.jpg}}
    \caption{Estado Final}
    \label{img5}
\end{figure}

\subsection{Fun\c c\~ao Utilidade}
O jogador ganha caso tenha mais sementes no seu depósito do que o seu adversário. A função utilidade simplesmente verifica quem ganhou a partida e retorna um valor positivo em caso de vitória, um valor negativo em caso de derrota e zero em caso de empate.

\begin{figure}[htbp]
    \centerline{\includegraphics[scale=0.3]{utilidade.jpg}}
    \caption{Imagem da função utilidade}
    \label{img6}
\end{figure}

\section{Trabalho Relacionado}

Na nossa pesquisa encontramos j\'a algum trabalho feito no que toca \`a implementa\c c\~ao deste jogo assim como na implementa\c c\~ao de intelig\^encia artificial\cite{b2}\cite{b3}\cite{b9}.
Tamb\'em nas aulas da disciplina foram implementados algoritmos id\^enticos ao deste projeto, pelo que certamente serve de fonte de auxílio.
Apesar de já existirem implementações do jogo mancala com o algoritmo minimax, o projeto do jogo quatro em linha referenciado consegue ser o mais completo, pois a forma como está estruturado separa muito bem a parte referente ao jogo da parte que diz respeito à IA.
Com esta estrutura conseguimos ter um excelente ponto de partida para o nosso projeto e ideia bem definida daquilo que tinhamos que fazer.

\section{Implementa\c c\~ao do Jogo}
Inserir Texto

\section{Algoritmos Implementados}
Inserir Texto

\section{Experiências e Resultados}
Inserir Texto

\section{Conclus\~ao}

Apesar de estes algoritmos j\'a terem sido abordados nas aulas, o desafio ser\'a juntar estes conhecimentos de forma a criar um projeto coeso e funcional.

% Before you begin to format your paper, first write and save the content as a 
% separate text file. Complete all content and organizational editing before 
% formatting. Please note sections \ref{AA}--\ref{SCM} below for more information on 
% proofreading, spelling and grammar.

% Keep your text and graphic files separate until after the text has been 
% formatted and styled. Do not number text heads---{\LaTeX} will do that 
% for you.

% \subsection{Abbreviations and Acronyms}\label{AA}
% Define abbreviations and acronyms the first time they are used in the text, 
% even after they have been defined in the abstract. Abbreviations such as 
% IEEE, SI, MKS, CGS, ac, dc, and rms do not have to be defined. Do not use 
% abbreviations in the title or heads unless they are unavoidable.

% \subsection{Units}
% \begin{itemize}
% \item Use either SI (MKS) or CGS as primary units. (SI units are encouraged.) English units may be used as secondary units (in parentheses). An exception would be the use of English units as identifiers in trade, such as ``3.5-inch disk drive''.
% \item Avoid combining SI and CGS units, such as current in amperes and magnetic field in oersteds. This often leads to confusion because equations do not balance dimensionally. If you must use mixed units, clearly state the units for each quantity that you use in an equation.
% \item Do not mix complete spellings and abbreviations of units: ``Wb/m\textsuperscript{2}'' or ``webers per square meter'', not ``webers/m\textsuperscript{2}''. Spell out units when they appear in text: ``. . . a few henries'', not ``. . . a few H''.
% \item Use a zero before decimal points: ``0.25'', not ``.25''. Use ``cm\textsuperscript{3}'', not ``cc''.)
% \end{itemize}

% \subsection{Equations}
% Number equations consecutively. To make your 
% equations more compact, you may use the solidus (~/~), the exp function, or 
% appropriate exponents. Italicize Roman symbols for quantities and variables, 
% but not Greek symbols. Use a long dash rather than a hyphen for a minus 
% sign. Punctuate equations with commas or periods when they are part of a 
% sentence, as in:
% \begin{equation}
% a+b=\gamma\label{eq}
% \end{equation}

% Be sure that the 
% symbols in your equation have been defined before or immediately following 
% the equation. Use ``\eqref{eq}'', not ``Eq.~\eqref{eq}'' or ``equation \eqref{eq}'', except at 
% the beginning of a sentence: ``Equation \eqref{eq} is . . .''

% \subsection{\LaTeX-Specific Advice}

% Please use ``soft'' (e.g., \verb|\eqref{Eq}|) cross references instead
% of ``hard'' references (e.g., \verb|(1)|). That will make it possible
% to combine sections, add equations, or change the order of figures or
% citations without having to go through the file line by line.

% Please don't use the \verb|{eqnarray}| equation environment. Use
% \verb|{align}| or \verb|{IEEEeqnarray}| instead. The \verb|{eqnarray}|
% environment leaves unsightly spaces around relation symbols.

% Please note that the \verb|{subequations}| environment in {\LaTeX}
% will increment the main equation counter even when there are no
% equation numbers displayed. If you forget that, you might write an
% article in which the equation numbers skip from (17) to (20), causing
% the copy editors to wonder if you've discovered a new method of
% counting.

% {\BibTeX} does not work by magic. It doesn't get the bibliographic
% data from thin air but from .bib files. If you use {\BibTeX} to produce a
% bibliography you must send the .bib files. 

% {\LaTeX} can't read your mind. If you assign the same label to a
% subsubsection and a table, you might find that Table I has been cross
% referenced as Table IV-B3. 

% {\LaTeX} does not have precognitive abilities. If you put a
% \verb|\label| command before the command that updates the counter it's
% supposed to be using, the label will pick up the last counter to be
% cross referenced instead. In particular, a \verb|\label| command
% should not go before the caption of a figure or a table.

% Do not use \verb|\nonumber| inside the \verb|{array}| environment. It
% will not stop equation numbers inside \verb|{array}| (there won't be
% any anyway) and it might stop a wanted equation number in the
% surrounding equation.

% \subsection{Some Common Mistakes}\label{SCM}
% \begin{itemize}
% \item The word ``data'' is plural, not singular.
% \item The subscript for the permeability of vacuum $\mu_{0}$, and other common scientific constants, is zero with subscript formatting, not a lowercase letter ``o''.
% \item In American English, commas, semicolons, periods, question and exclamation marks are located within quotation marks only when a complete thought or name is cited, such as a title or full quotation. When quotation marks are used, instead of a bold or italic typeface, to highlight a word or phrase, punctuation should appear outside of the quotation marks. A parenthetical phrase or statement at the end of a sentence is punctuated outside of the closing parenthesis (like this). (A parenthetical sentence is punctuated within the parentheses.)
% \item A graph within a graph is an ``inset'', not an ``insert''. The word alternatively is preferred to the word ``alternately'' (unless you really mean something that alternates).
% \item Do not use the word ``essentially'' to mean ``approximately'' or ``effectively''.
% \item In your paper title, if the words ``that uses'' can accurately replace the word ``using'', capitalize the ``u''; if not, keep using lower-cased.
% \item Be aware of the different meanings of the homophones ``affect'' and ``effect'', ``complement'' and ``compliment'', ``discreet'' and ``discrete'', ``principal'' and ``principle''.
% \item Do not confuse ``imply'' and ``infer''.
% \item The prefix ``non'' is not a word; it should be joined to the word it modifies, usually without a hyphen.
% \item There is no period after the ``et'' in the Latin abbreviation ``et al.''.
% \item The abbreviation ``i.e.'' means ``that is'', and the abbreviation ``e.g.'' means ``for example''.
% \end{itemize}
% An excellent style manual for science writers is \cite{b7}.

% \subsection{Authors and Affiliations}
% \textbf{The class file is designed for, but not limited to, six authors.} A 
% minimum of one author is required for all conference articles. Author names 
% should be listed starting from left to right and then moving down to the 
% next line. This is the author sequence that will be used in future citations 
% and by indexing services. Names should not be listed in columns nor group by 
% affiliation. Please keep your affiliations as succinct as possible (for 
% example, do not differentiate among departments of the same organization).

% \subsection{Identify the Headings}
% Headings, or heads, are organizational devices that guide the reader through 
% your paper. There are two types: component heads and text heads.

% Component heads identify the different components of your paper and are not 
% topically subordinate to each other. Examples include Acknowledgments and 
% References and, for these, the correct style to use is ``Heading 5''. Use 
% ``figure caption'' for your Figure captions, and ``table head'' for your 
% table title. Run-in heads, such as ``Abstract'', will require you to apply a 
% style (in this case, italic) in addition to the style provided by the drop 
% down menu to differentiate the head from the text.

% Text heads organize the topics on a relational, hierarchical basis. For 
% example, the paper title is the primary text head because all subsequent 
% material relates and elaborates on this one topic. If there are two or more 
% sub-topics, the next level head (uppercase Roman numerals) should be used 
% and, conversely, if there are not at least two sub-topics, then no subheads 
% should be introduced.

% \subsection{Figures and Tables}
% \paragraph{Positioning Figures and Tables} Place figures and tables at the top and 
% bottom of columns. Avoid placing them in the middle of columns. Large 
% figures and tables may span across both columns. Figure captions should be 
% below the figures; table heads should appear above the tables. Insert 
% figures and tables after they are cited in the text. Use the abbreviation 
% ``Fig.~\ref{fig}'', even at the beginning of a sentence.

% \begin{table}[htbp]
% \caption{Table Type Styles}
% \begin{center}
% \begin{tabular}{|c|c|c|c|}
% \hline
% \textbf{Table}&\multicolumn{3}{|c|}{\textbf{Table Column Head}} \\
% \cline{2-4} 
% \textbf{Head} & \textbf{\textit{Table column subhead}}& \textbf{\textit{Subhead}}& \textbf{\textit{Subhead}} \\
% \hline
% copy& More table copy$^{\mathrm{a}}$& &  \\
% \hline
% \multicolumn{4}{l}{$^{\mathrm{a}}$Sample of a Table footnote.}
% \end{tabular}
% \label{tab1}
% \end{center}
% \end{table}

% \begin{figure}[htbp]
% \centerline{\includegraphics{fig1.png}}
% \caption{Example of a figure caption.}
% \label{fig}
% \end{figure}

% Figure Labels: Use 8 point Times New Roman for Figure labels. Use words 
% rather than symbols or abbreviations when writing Figure axis labels to 
% avoid confusing the reader. As an example, write the quantity 
% ``Magnetization'', or ``Magnetization, M'', not just ``M''. If including 
% units in the label, present them within parentheses. Do not label axes only 
% with units. In the example, write ``Magnetization (A/m)'' or ``Magnetization 
% \{A[m(1)]\}'', not just ``A/m''. Do not label axes with a ratio of 
% quantities and units. For example, write ``Temperature (K)'', not 
% ``Temperature/K''.

% \section*{Acknowledgment}

% The preferred spelling of the word ``acknowledgment'' in America is without 
% an ``e'' after the ``g''. Avoid the stilted expression ``one of us (R. B. 
% G.) thanks $\ldots$''. Instead, try ``R. B. G. thanks$\ldots$''. Put sponsor 
% acknowledgments in the unnumbered footnote on the first page.

% \section*{References}

% Please number citations consecutively within brackets \cite{b1}. The 
% sentence punctuation follows the bracket \cite{b2}. Refer simply to the reference 
% number, as in \cite{b3}---do not use ``Ref. \cite{b3}'' or ``reference \cite{b3}'' except at 
% the beginning of a sentence: ``Reference \cite{b3} was the first $\ldots$''

% Number footnotes separately in superscripts. Place the actual footnote at 
% the bottom of the column in which it was cited. Do not put footnotes in the 
% abstract or reference list. Use letters for table footnotes.

% Unless there are six authors or more give all authors' names; do not use 
% ``et al.''. Papers that have not been published, even if they have been 
% submitted for publication, should be cited as ``unpublished'' \cite{b4}. Papers 
% that have been accepted for publication should be cited as ``in press'' \cite{b5}. 
% Capitalize only the first word in a paper title, except for proper nouns and 
% element symbols.

% For papers published in translation journals, please give the English 
% citation first, followed by the original foreign-language citation \cite{b6}.

\begin{thebibliography}{00}
\bibitem{b1} Erik Arneson, https://www.thesprucecrafts.com/how-to-play-mancala-409424, 3/05/2019
\bibitem{b2} https://github.com/mhchong/Mancala-Game
\bibitem{b3} https://github.com/cypreess/py-mancala
\bibitem{b4} https://www.johnpratt.com/items/mancala/index.html
\bibitem{b5} https://en.wikipedia.org/wiki/Minimax#Pseudocode
\bibitem{b6} https://en.wikipedia.org/wiki/Alpha–beta\_pruning#Pseudocode
\bibitem{b7} https://www.ultraboardgames.com/mancala/strategies.php
\bibitem{b8} https://www.ultraboardgames.com/mancala/best-opening-move.php
\bibitem{b9} https://github.com/KeithGalli/Connect4-Python
\end{thebibliography}
% \vspace{12pt}
% \color{red}
% IEEE conference templates contain guidance text for composing and formatting conference papers. Please ensure that all template text is removed from your conference paper prior to submission to the conference. Failure to remove the template text from your paper may result in your paper not being published.

\end{document}